% \documentclass[10pt]{beamer}
\documentclass[handout]{beamer}

\usetheme[progressbar=frametitle]{metropolis}
\usepackage{appendixnumberbeamer}
\setbeamercolor{background canvas}{bg=white}

\usepackage{booktabs}
\usepackage[scale=2]{ccicons}

\usepackage{pgfplots}
\usepackage{svg}
\usepgfplotslibrary{dateplot}

\usepackage{xspace}
\newcommand{\themename}{\textbf{\textsc{metropolis}}\xspace}

\usepackage{caption}
\usepackage{subcaption}
\captionsetup{font=scriptsize,labelfont=scriptsize}

\usepackage{adjustbox}

% -------symbols--------
\definecolor{cr}{RGB}{230, 92, 0}
\definecolor{red}{RGB}{255, 0, 0}
\newcommand{\ca}[1]{\textcolor{cr}{#1}}
\newcommand{\re}[1]{\textcolor{red}{#1}}

\begin{document}

\begin{frame}
    \centering
    \includegraphics[width=1\textwidth]{title.png}
\end{frame}

\begin{frame}[<+->]{Motivação}
    \begin{itemize}
    \setlength\itemsep{1em}
        \item Salas cirúrgicas representam > 30\% dos custos hospitalares
        \item Incerteza nas durações de cirurgias gera:
            \begin{itemize}
                \item Horas extras
                \item Tempo ocioso
                \item Perda de receita
            \end{itemize}
        \item Necessidade de maior previsibilidade
    \end{itemize}
\end{frame}


\begin{frame}[<+->]{Objetivo}
Propor um modelo que reduza a variabilidade dos custos operacionais sob incerteza:
    \begin{itemize}
        \setlength\itemsep{1em}
        \item Minimizar custos de \ca{horas extras} e \ca{ociosidade} de salas
        \item Melhorar desempenho em cenários críticos
    \end{itemize}
\end{frame}


\begin{frame}[<+->]{Problema Estudado}
    \begin{itemize}
    \setlength\itemsep{1em}
    \item Conjunto de pacientes eletivos
    \item Conjunto de salas cirúrgicas
    \item Durações cirúrgicas não determinísticas
    \item \ca{Decisões}:
        \begin{itemize}
            \item Alocação às salas
            \item Sequenciamento cirúrgico
            \item Tempos de início (usado somente para avaliar o custo)
        \end{itemize}
        \item Ou seja, \ca{dadas as durações cirúrgicas, como alocar e sequenciar pacientes nas salas?}
    \end{itemize}
\end{frame}


\begin{frame}[<+->]{Incerteza nas Durações}
    \begin{itemize}
        \setlength\itemsep{1em}
        \item Distribuições probabilísticas (assumem lognormal)
        \item Monte Carlo para gerar \ca{cenários}
        \begin{itemize}
            \item Cada cenário = uma possível realização (sorteio) das durações dos pacientes
            \item Exemplo: 100 cenários simulados
        \end{itemize}
    \end{itemize}
\end{frame}


\begin{frame}[<+->]{Otimização por Valor Esperado (EV): Limitação}
    \begin{itemize}
        \setlength\itemsep{1em}
        \item Modelo ``neutro ao risco''
        \item Apenas minimiza valor esperado do custo (média ponderada)
        \item Ignora variabilidade e extremos
    \end{itemize}
\end{frame}


\begin{frame}[<+->]{O que é VaR?}
    \begin{itemize}
        \setlength\itemsep{1em}
        \item \ca{Value-at-Risk}
        \item Usado em finanças para medir risco de perdas
        \item Limiar que cobre \textit{$\alpha$} da distribuição
        \item Exemplo: 
        \begin{itemize}
            \item Distribuição de durações de cirurgias: $[30, 35, 60, 70, 85, 87, 89, 91, 95, 121]$
            \item VaR 80\% ($\alpha=0.8$) = 91 minutos
            \item 20\% das durações excedem esse valor
        \end{itemize}
        \item Limitação: não informa sobre a gravidade das perdas além do VaR
    \end{itemize}
\end{frame}


\begin{frame}[<+->]{O que é CVaR?}
    \begin{itemize}
        \setlength\itemsep{1em}
        \item \ca{Conditional Value-at-Risk}
        \item Custo médio dos piores cenários (cauda da distribuição)
        \item Foca em minimizar perdas extremas
        \item Exemplo:
        \begin{itemize}
            \item Usando o exemplo anterior com VaR 80\% = 91 minutos
            \item CVaR 80\% = ~108 minutos
            \item i.e., a média das durações que excedem 91 minutos é ~108 minutos
        \end{itemize}
    \end{itemize}
\end{frame}


\begin{frame}[<+->]{Modelo CVaR}
    \begin{itemize}
        \setlength\itemsep{1em}
        \item Formulação SMILP (Stochastic MILP)
        \item Variável $\zeta$: aproximador de VaR
        \item Minimiza custo esperado acima de $\zeta$
    \end{itemize}
\end{frame}


\begin{frame}[<+->]{Modelo CVaR}
    \begin{columns}
        \begin{column}{0.5\textwidth}
            \centering
            \includegraphics[width=1.1\textwidth]{symb_1.png}
        \end{column}
        \begin{column}{0.5\textwidth}
            \centering
            \includegraphics[width=1.1\textwidth]{symb_2.png}
        \end{column}
    \end{columns}
\end{frame}


\begin{frame}[<+->]{Modelo CVaR}
    \centering
    \includegraphics[width=.8\textwidth]{M1.png}
\end{frame}


\begin{frame}[<+->]{Modelo CVaR}
    \centering
    \includegraphics[width=.8\textwidth]{M2.png}
\end{frame}


\begin{frame}[<+->]{Experimentos Computacionais}
    \action{Parâmetros:}
    \begin{center}
        \includegraphics[width=0.8\textwidth]{parametros.png}
    \end{center}
    \bigskip

    \action{Comparação de custos associados entre as abordagens:}
    \begin{itemize}
        \setlength\itemsep{1em}
        \item Modelo por Valor Esperado (EV) -- só definir o $\alpha = 0$ na formulação CVaR
        \item Modelo CVaR
    \end{itemize}
\end{frame}


\begin{frame}[<+->]{Vetor de Utilização ($\tau$)}
    \begin{itemize}
        \item Um único planejamento é gerado pelo modelo (alocação + sequência)
        \item Os tempos de cirurgia variam em cada cenário
        \item Para cada cenário, calcula-se:
        \begin{itemize}
            \item tempo extras das salas (multiplicado por um custo)
            \item tempo ocioso das salas (multiplicado por um custo)
        \end{itemize}
        \item A soma desses valores em cada cenário forma um vetor:
        \[
        \tau = [\tau_1, \tau_2, \ldots, \tau_{|S|}]
        \]
        \item O vetor $\tau$ representa como o plano se comporta em todos os futuros possíveis
        \item Então os autores calculam a média ($\mu_{CVaR}$) e o desvio padrão ($\sigma_{CVaR}$) desse vetor para análise
    \end{itemize}
\end{frame}


\begin{frame}[<+->]{Resultados}
    \begin{center}
        \includegraphics[width=0.8\textwidth]{results_1.png}
    \end{center}
    \bigskip

    \action{Quanto maior o $\alpha$, menor a variabilidade e maior o custo médio }
\end{frame}


\begin{frame}[<+->]{Melhoria em Piores Cenários}
    \begin{itemize}
        \setlength\itemsep{1em}
        \item Redução \textbf{~9\%} no pior caso
        \item 2.5\% a 4\% de economia concentrada nos piores cenários
    \end{itemize}
\end{frame}


\begin{frame}[<+->]{Complexidade Computacional}
    \begin{center}
        \includegraphics[width=0.9\textwidth]{table2.png}
    \end{center}
    \bigskip
    \begin{center}
        \includegraphics[width=0.9\textwidth]{table3.png}
    \end{center}

\end{frame}


\begin{frame}[<+->]{Contribuições Científicas}
    \begin{itemize}
        \setlength\itemsep{1em}
        \item 1ª aplicação de CVaR em OR scheduling \textit{(não mais...)}
        \item Melhor desempenho sob incerteza
        \item Flexibilidade de decisão baseada em risco
        \item CVaR → escalonamentos mais estáveis e realistas
        \item Reduz a imprevisibilidade operacional
    \end{itemize}
\end{frame}

\end{document}